\documentclass[../main.tex]{subfiles}

\begin{document}
\section{Primers, plasmids and Strains}
\label{sec:plaspri}

\subsection{Primers}
\label{sec:plaspri:pri}
\begin{table}[h!]
\begin{center}
{\footnotesize
\begin{tabular}{c|l|p{4cm}|c|c|p{5cm}}
\textbf{Primer}
	&\textbf{Name}
	&\textbf{Sequence}
	&\textbf{Length}
	&\textbf{Tm\linebreak(\si{\degree}C})\myfootnotemark
	&\textbf{Description}
	\\\hline
\textbf{WAC1}
	&eYFP A206K
	&\dna{TACCTGAGCTACCAGTCCAAACTGAGCAAAGACCCCAAC}
	&39
	&81
	&A206K mutation in eYFP
	\\
\textbf{WAC2}
	&eYFP A206K anti
	&\dna{GTTGGGGTCTTTGCTCAGTTTGGACTGGTAGCTCAGGTA}
	&39
	&81
	&A206K mutation in eYFP
	\\
\textbf{WAC5}
	&mVenusF	
	&\dna{tagct{\underline{ggagctcaagctt}}ATGGTGAGCAAGGGCGAGG}
	&37
	&83
	&Extract mVenus and add SacI and HindIII restriction sites
	\\
\textbf{WAC6}
	&mVenusR
	&\dna{aggtCT{\underline{TGTACA}}GCTCGTCCATGCCG}
	&26
	&75
	&Extract mVenus to include BsrGI
	\\
\textbf{WAC7}
	&pBAD33-split-ccw
	&\dna{GGACAGCTGATAGAAACAGAAGC}
	&23
	&63
	&Split pBAD33 in counter-clockwise direction
	\\
\textbf{WAC8}
	&pBAD33-split-cw
	&\dna{TTTTTGAGGTGCTCCAGTGG}
	&20
	&66
	&Split pBAD33 in clockwise direction
	\\
\textbf{WAC9}
	&pTRC99a-split-ccw
	&\dna{TTCCTCGCTCACTGACTCGC}
	&20
	&69
	&Split pTRC99a in counter-clockwise direction
	\\
\textbf{WAC10}
	&pTRC99a-split-cw
	&\dna{GCCGAACGACCGAGCG}
	&16
	&70
	&Split pTRC99a in clockwise direction
	\\
\textbf{WAC11}
	&e*FP-vector-ccw
	&\dna{CCACCCCGGTGAACAGC}
	&17
	&68
	&Delete e*FP\myfootnotemark~from plasmid leaving overhang
	\\
\textbf{WAC12}
	&e*FP-vector-cw
	&\dna{TCCTGCTGGAGTTCGTGACC}
	&20
	&68
	&Complementary to \textbf{WAC11}
	\\
\textbf{WAC13}
	&e*FP-insert-ccw
	&\dna{GTCCATGCCGAGAGTGATCC}
	&20
	&67
	&Extract most of e*FP with correct overhangs for use with \textbf{WAC11} and \textbf{WAC12}
	\\
\textbf{WAC14}
	&e*FP-insert-cw
	&\dna{GGTGAGCAAGGGCGAGG}
	&17
	&68
	&Complementary to \textbf{WAC14}
\end{tabular}
}
\caption{Primers used in this report. In the sequences, upper case letters correspond to complementary sequences to the template DNA. Lower case letters correspond to extensions, and underlined portions highlight restriction sites.}
\label{tbl:primers}
\end{center}
\end{table}
\myfootnotetext{\url{ http://www.finnzymes.fi/tm_determination.html } using modified Breslaur}
\myfootnotetext{e*FP refers to any GFP derivative containing no mutations between nucleotide 20 and nucleotide 685.}
The primers \textbf{WAC7} through \textbf{WAC14} comprise a toolkit to exchange any one GFP derivative with any other GFP derivative (subject to the constraints described in table~\ref{tbl:primers}). \textbf{WAC13} and \textbf{WAC14} cut out and amplify the new fluorescent protein. \textbf{WAC11} and \textbf{WAC12} cut out and discard the old fluorescent protein, amplifying whatever is left. \textbf{WAC7} and \textbf{8} and \textbf{WAC9} and \textbf{10} split pBAD33 and pTRC99a respectively about halfway round, to reduce the maximum run length for polymerase in the PCR.

As an example, to replace the GFP tagging a protein in pBAD33 with YFP, the three following PCR reactions must be done:
\begin{center}
\begin{tabular}{lcc}
\textit{Template}&\multicolumn{2}{c}{\textit{Primers}} \\
GFP-protein-pBAD33	&	\textbf{WAC7} 	&	\textbf{WAC12}\\
GFP-protein-pBAD33	&	\textbf{WAC8} 	&	\textbf{WAC11}\\
YFP					&	\textbf{WAC13}	&	\textbf{WAC14}
\end{tabular}
\end{center}


The three products of these can then be put into a Gibson Assembly reaction to obtain the required end product.
\subsection{Plasmids}
\label{sec:plaspri:plas}
~
\begin{table}[h!]
\begin{center}
\plasmidtable{plaspri/myplasmids}
\caption{Plasmids created during the course of this project. All are derivatives of plasmids from Victor Sourjik (table~\ref{tbl:vsplasmids})}
\label{tbl:myplasmids}
\end{center}
\end{table}

\begin{table}[h!]
\begin{center}
\plasmidtable{plaspri/vsplasmids}
\caption{Plasmids from Viktor Sourjik}
\label{tbl:vsplasmids}
\end{center}
\end{table}

\subsection{Strains}
~
\begin{table}[h!]
\begin{center}
\straintable{plaspri/klstrains}
\caption{Strains used in the project}
\label{tbl:klstrains}
\end{center}
\end{table}

\begin{table}[h!]
\begin{center}
\straintable{plaspri/newstrains}
\caption{Strains created during the course of this project}
\label{tbl:newstrains}
\end{center}
\end{table}

\end{document}