\documentclass[../main.tex]{subfiles}

\begin{document}

\section{Discussion}

\subsection{Microscope Development}

\subsubsection{TIRF Microscopy}
TIRF microscopy systems were demonstrated by Nikon and Leica. The position and direction of the laser for TIRF illumination is highly dependent on a number of parameters, including the wavelength, the material properties of the oil and coverslip, and the mounting media itself. Whilst the Nikon system required manual calibration each time the laser was disrupted, the Leica system had pre-defined settings to allow the user to request a penetration depth for the evanescent wave and the microscope to be set accordingly.

Manual calibration of TIRF is achieved by shifting the laser across the back focal plane of the microscope. As it leaves the edge it will form a crescent of illumination around one side of the back focal plane. In the image plane this should result in a very high contrast image with only cells extremely close to the coverslip illuminated. Unfortunately, in the Nikon we were unable to obtain both of these. When the laser was forming a crescent, the image was poor; when the image was high contrast, the laser was not forming a crescent.

We had in fact achieved oblique fluorescence microscopy, where the laser enters at a steep angle just short of total internal reflectance. This penetrates to a much greater depth than TIRF, but does not illuminate as much of the background as epifluorescence, resulting in a much better contrast to background.

We were not able to check the results in the Nikon against those in the Leica; we did doubt whether we had achieved TIRF in Leica either. However, after our experience with the Nikon system we decided that it would be unwise to pursue TIRF further for two main reasons:

\begin{enumerate}
	\item Our work is highly quantitative. From what we had seen in the demonstrations, TIRF had several properties that could make it less quantitative than required, such as uneven penetration depth and illumination, and a highly complex and hard to reproduce set up procedure.
	\item The clusters can be located, and indeed move around, anywhere on the polar hemisphere of the cell. An illumination penetration depth of only \SI{200}{\nano\meter} will mean that a lot of clusters may not be seen, whatever the orientation of the cell. 
\end{enumerate}

\subsubsection{Super-resolution microscopy}

\paragraph{STED microscopy} Whilst an improvement on confocal in terms of resolution, it did not offer significant enough improvement for our purposes to give any tangible benefit, whilst also being hugely more expensive.

\paragraph{Localisation Microscopy} We were only able to obtain a brief demo of a Leica GSD system. However, the system worked extremely well on KL1005 cells without any special preparation. However, there is still some question as to whether localisation microscopy could provide the required resolution for our purposes. Also, even with the fastest cameras on the market, the time resolution would not be adequate for the live imaging that we have planned. However, the technique and technologies are still improving, and so in the future this could be a viable and beneficial method to employ.

\subsubsection{Final Microscope Setup}

We agreed on our final choice of microscope after trying many different combinations. The Nikon was chosen for its high degree of modularity, giving us plenty of room to expand when necessary. The 100X TIRF lens is one of the best in its class, and the 40X has a low numerical aperture to make it suitable for anisotropy measurements~\citep{lakowicz}. The Andor camera met the right balance of sensor size and speed - a large sensor both costs more and takes longer to read out. The field of view of the chosen camera was around \SI{82}{\micro\meter} on a 100X objective; more than enough to view objects not usually more than \SI{2.5}{\micro\meter} long. The Andor camera also had very high sensitivity, which was critical for our application due to the rapid imaging required and low intensity of emission of our cells.

A LED lighting system was used primarily as the easiest way to mix two wavelengths of light without using lasers, as required for blue light repellent with imaging. As a side bonus, LEDs have better stability, are cheaper to install and maintain, and do not require mechanical shutters to do extremely fast switching times. 

\subsection{Cluster Sizing using Image Processing}

Results from the epifluoresence microscopy (section~\ref{sec:results:cs:epi}) are very inadequate, due to a very low cell count. The experiments were mostly useful for developing and testing the image processing scripts, to maximise speed and accuracy.

The results from the confocal microscopy (section~\ref{sec:results:cs:confocal}) are much more encouraging. Thanks to a much higher cell count the results are far more informative, showing a small but definite increase in CheZ clustering, with a p value of 0.0349\%. More repeats need to be done to verify this result, but the preliminary indication is that the experiments support the hypothesis of dynamic CheZ clustering.

\subsection{Gibson Assembly}
Gibson Assembly proved to be a enormously efficient technique, with all cells successfully sequenced showing the desired changes. Over the course of the project the Gibson Assembly reaction size was reduced from \SI{20}{\micro\litre} to \SI{10}{\micro\litre} without any observable decrease in transformation efficiency. 

\subsection{Inducer calibration}

Due to high cell movement most of the images suffer from motion blur. However, there are enough good cells to see a marked increase in contrast in each increase in inducer concentration, indicating that protein expression has increased correspondingly and not reached saturation.

\end{document}