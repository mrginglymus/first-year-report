\documentclass[../main.tex]{subfiles}

\begin{document}

\section{Discussion}

\subsection{Microscope Development}

\subsubsection{TIRF Microscopy}
TIRF microscopy systems were demonstrated by Nikon and Leica. The position and direction of the laser for TIRF illumination is highly dependent on a number of parameters, including the wavelength, the material properties of the oil and coverslip, and the mounting media itself. Whilst the Nikon system required manual calibration each time the laser was disrupted, the Leica system had pre-defined settings to allow the user to request a penetration depth for the evanescent wave and the microscope to be set accordingly.

Manual calibration of TIRF is achieved by shifting the laser across the back focal plane of the microscope. As it leaves the edge it will form a crescent of illumination around one side of the back focal plane. In the image plane this should result in a very high contrast image with only cells extremely close to the coverslip illuminated. Unfortunately, in the Nikon we were unable to obtain both of these. When the laser was forming a crescent, the image was poor; when the image was high contrast, the laser was not forming a crescent.

We had in fact achieved oblique fluorescence microscopy, where the laser enters at a steep angle just short of total internal reflectance. This penetrates to a much greater depth than TIRF, but does not illuminate as much of the background as epifluorescence, resulting in a much better contrast to background.

We were not able to check the results in the Nikon against those in the Leica due to timing issues; we did doubt whether we had achieved TIRF in Leica either. However, after our experience with the Nikon system we decided that it would be unwise to pursue TIRF further:

\begin{itemize}
	\item Due to the difficulty in setting up TIRF it would be difficult to accurately obtain the same penetration depth each experiment
	\item The penetration depth of the evanescent wave is a function of what it encounters - this will give uneven penetration across the image
	\item The power of the evanescent wave decays rapidly over the distance it penetrates, giving further rise to uneven illumination
	\item Ensuring that the polar clusters are within the evanescent wave proved troublesome, as the wave only penetrated a few hundred nanometers, whilst the cells were around one micron across. Assuming the cell was lying flat, the cluster could be well out of reach of the wave. We did investigate the possibility of forcing the cells into standing head on, but as this is not a standard technique it would have distracted from the aims of the experiments
	\item Perhaps most significantly, we were interested in the fraction of proteins in the cluster to those in the cell, and this required illumination of the entire cell. This would be highly unworkable if the system required recalibration between each method of illumination
\end{itemize}

\subsubsection{Super-resolution microscopy}

\paragraph{STED microscopy} Whilst an improvement on confocal in terms of resolution, it did not offer significant enough improvement for our purposes to give any tangible benefit, whilst also being hugely more expensive.

\paragraph{STORM/PALM} We were only able to obtain a brief demo of a Leica PALM system. However, the system worked extremely well on KL1005 cells without any special preparation. Unfortunately, there is little evidence yet that STORM/PALM like systems are suitable for our purposes. Despite being considerably cheaper than confocal microscopes, we still considered it too much of a risk to invest serious time or money in presently.

\subsubsection{Final Microscope Setup}

We agreed on our final choice of microscope after trying many different combinations. The Nikon was chosen for its high degree of modularity, giving us plenty of room to expand when necessary. The 100X TIRF lens is one of the best in its class, and the 40X has a low numerical aperture to make it suitable for anisotropy measurements~\citep{lakowicz}. The Andor camera met the right balance of sensor size and speed - a large sensor both costs more and takes longer to read out. The field of view of the chosen camera was around \SI{82}{\micro\meter} on a 100X objective; more than enough to view objects not usually more than \SI{1.5}{\micro\meter} long. The Andor camera also had very high sensitivity, which was critical for our application due to the rapid imaging required and low intensity of emission of our cells.

\subsection{Cluster Sizing using Image Processing}

Results from the epifluoresence microscopy (section~\ref{sec:results:cs:epi}) are very inadequate, due to a very low cell count. The experiments were mostly useful for developing and testing the image processing scripts, to maximise speed and accuracy.

The results from the confocal microscopy (section~\ref{sec:results:cs:confocal}) are much more encouraging. Thanks to a much higher cell count the results are far more informative, showing a small but definite increase in CheZ clustering, with a p value of 0.0349\%. More repeats need to be done to verify this result, but the preliminary indication is that the experiments support the hypothesis of dynamic CheZ clustering.

\subsection{Gibson Assembly}
Gibson Assembly proved to be a enormously efficient technique, with all cells successfully sequenced showing the desired changes. Over the course of the project the Gibson Assembly reaction size was reduced from \SI{20}{\micro\litre} to \SI{10}{\micro\litre} without any observable decrease in transformation efficiency. 

\subsection{Inducer calibration}

Due to high cell movement most of the images suffer from motion blur. However, there are enough good cells to see a marked increase in contrast in each increase in inducer concentration, indicating that protein expression has increased correspondingly and not reached saturation.

\end{document}