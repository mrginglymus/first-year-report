\documentclass[12pt]{article}
\linespread{1.3}

\newcommand{\docroot}{/Users/Bill/Dropbox/Work/PhD/reports/first-year-report/}

\setcounter{tocdepth}{1}

\usepackage[parfill]{parskip}
\usepackage{natbib}			% bibliography with nature style
\usepackage{subfiles}						% for including files but compiling separately when needed
\usepackage[alsoload=synchem]{siunitx}		% just nice
\usepackage{gensymb}							% for degree symbol, probably others
\usepackage[version=3]{mhchem}				% for chemical equations
\usepackage{multirow}						% for multi row/columns in tables
\usepackage{fixltx2e}						% subscript and superscript in text
\usepackage{hyperref}						% for urls
\usepackage{seqsplit}						% for inserting linebreaks into DNA sequences 
\usepackage[top=2.5cm, bottom=2.5cm, left=2.5cm, right=2.5cm]{geometry}                % See geometry.pdf to learn the layout options. There are lots.
\usepackage[font={small,sf}, labelfont=bf]{subfig}
\usepackage{threeparttable}
\usepackage{datatool}
\usepackage{lipsum}
\usepackage{doi}
\usepackage{ordinalref}
\usepackage[pdftex]{graphicx}
\usepackage{epstopdf}
\usepackage{xspace}
\usepackage[format=hang, justification=raggedright, singlelinecheck=false, font={small, sf}, labelfont=bf]{caption}
\usepackage[usenames,dvipsnames]{xcolor}
\input{\docroot rgb.tex}


\renewcommand*\abstractname{Summary}
\newcounter{myfootertablecounter}

\newcommand\myfootnotemark{%
%\refstepcounter{footnote}%
\addtocounter{footnote}{1}%
\footnotemark[\thefootnote]%
}%

\newcommand\myfootnotemarkhold{%
\footnotemark[\thefootnote]%
}%

\newcommand\myfootnotetext[1]{%
\addtocounter{myfootertablecounter}{1}
\footnotetext[\value{myfootertablecounter}]{#1}
}

% from now on, myfootnote has to be used rather than footnote to
% adapt the myfootercounter
\newcommand\myfootnote[1]{%
\addtocounter{myfootertablecounter}{1}
\footnote{#1}
}%


\DeclareSIUnit\Molar{\textsc{M}}				% molar unit definition
\DeclareSIUnit\base{bp}						% basepair unit definition
\DeclareSIUnit\unit{U}						% enzyme unit definition

\newcommand{\dna}[1]{\texttt{\seqsplit{#1}}}	% DNA monotype

\newcommand{\ecoli}{\textit{E. coli}\xspace}			% E. coli
\newcommand{\ecolilong}{\textit{Escherichia coli}\xspace}		% escherichia coli

\newcommand{\plasmidtable}[1]{%
\DTLloaddb{#1}{\docroot #1.csv}
\begin{tabular}{p{3cm}p{4cm}lcc}
\bfseries Name &
\bfseries Gene &
\bfseries Backbone &
\bfseries Antibiotic &
\bfseries Inducer
\DTLforeach{#1}{\plasmidname=Name,\gene=Gene,\backbone=Backbone,\antibiotic=Antibiotic,\inducer=Inducer}{\\\textbf{\plasmidname} & \textit{\gene} & \backbone & \antibiotic & \inducer}
\end{tabular}}

\newcommand{\straintable}[1]{%
\DTLloaddb{#1}{\docroot #1.csv}
\begin{tabular}{p{2cm}p{2cm}p{1.5cm}p{1.5cm}l}
\bfseries Name &
\bfseries Strain &
\multicolumn{2}{c}{\bfseries Plasmid(s)} &
\bfseries Genotype
\DTLforeach{#1}{\strainname=Name, \strain=Strain, \plasmidi=PlasmidI, \plasmidii=PlasmidII, \genotype=Genotype}
{\\\textbf{\strainname} & \strain & \plasmidi & \plasmidii & \textit{\genotype} }
\end{tabular}}

%\bibliography{\docroot refs}

\title{Development of Advanced Microscopy Techniques for Live Imaging of Dynamically Forming Chemotaxis Protein Clusters in \ecoli}
\date{June 2012}
\author{William A Collins\\\\\textsc{Supervisor}: Dr Karen Lipkow\\\\Cambridge Systems Biology Centre\\Department of Biochemistry}

\begin{document}

\pagenumbering{roman}
\setcounter{page}{1}


\maketitle
\newpage
\tableofcontents
\newpage
\addcontentsline{toc}{section}{\numberline {}List of Figures}
\listoffigures
\addcontentsline{toc}{section}{\numberline {}List of Tables}
\listoftables
\cleardoublepage

%TC:ignore
\addcontentsline{toc}{section}{\numberline {}Summary}
\begin{abstract}
The chemotaxis system in \ecoli is arguably the best understood signalling pathways in all biology, and is used as a springboard for research into similar pathways in other systems. Whilst much of the stoichiometry and kinetics of the pathway have been well established, the localisation of many of the proteins remains relatively unknown, yet has been shown through computer simulations to have a crucial influence on the dynamics of the system. Through advanced microscopy techniques and image processing we intend to prove the validity of some of these models and observe chemotaxis protein localisation in real time.


The proposed experiments require the generation of new strains of bacteria. This was done with the pioneering new technique for DNA manipulation called Gibson Assembly, removing the need for multifarious restriction enzymes but requiring the complex design of primers.


We show here the image processing and microscopy set-up developed to meet the challenging requirements of real-time protein localisation. Also presented are some of the early results from imaging, showing tentative support for the proposed models. For future work we consider the prospects for automating and optimising primers, specifically for use in Gibson Assembly but also more generally for conventional PCR.
\end{abstract}
\newpage

\pagenumbering{arabic}
\setcounter{page}{1}

%TC:endignore
%TC:fileinclude \subfile 1
\subfile{introduction/introduction}
\cleardoublepage

\subfile{matmeth/matmeth2}
\cleardoublepage

\subfile{results/results}
\cleardoublepage

\subfile{discussion/discussion}
\cleardoublepage

\subfile{future/future}
\cleardoublepage
%TC:ignore
\appendix
\subfile{plaspri/plaspri}
\cleardoublepage

\subfile{scripts/scripts}
\cleardoublepage

\subfile{acknowledgements/acknowledgements}
\cleardoublepage

\addcontentsline{toc}{section}{\numberline {}References}
\bibliography{\docroot refs}
\bibliographystyle{bill}
%\printbibliography
%TC:endignore
\end{document}